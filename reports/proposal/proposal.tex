\documentclass[11pt]{article}
\usepackage{acl2012}
\usepackage{geometry}
\usepackage{times}
\usepackage{latexsym}
\usepackage{amsmath}
\usepackage{multirow}
\usepackage{url}
\newgeometry{margin=2.85cm}
\DeclareMathOperator*{\argmax}{arg\,max}
\setlength\titlebox{6.8cm}    % Expanding the titlebox

\title{{\small CS224W Final Project} \\ Project Proposal}
\author{Thomas Dimson \\
  {\tt tdimson@cs.stanford.edu}
  \\\And
  Milind Ganjoo \\
  {\tt mganjoo@stanford.edu}
}
\date{}

\newcommand{\titlecite}[2]{``#1''~\cite{#2}}

\begin{document}
\maketitle

\section{Introduction}
Our project blah blah blah

\section{Reaction Paper}
Summary, critique, brainstorming.


\subsection{Quantifying Influence on Twitter}
In \titlecite{Everyone's an influencer: quantifying influence on twitter}{bakshy2011everyone}, 
some stuff happens.

% ROUGH NOTES
% * Paper observes cascade / diffusion events to quantify "influencers"
% * Defines influencers as those who disproportionally impact the spread of information
% ** We could investigate whether this corresponds to karma, if our network has cascades
% * Methodology: observe the origin of URLs, then track them as they are tweeted by followers
% * Trains model for cascades based on followers, tweets,  friends, past influence. Past influence most predictive
% ** Finds local features (direct followers) are more important than global. Could try to see if karma is influenced
%    by people close by
% * Also tries to find out the role of /content/ in cascades by using mech turk to classify the URLs
% ** Did not get good results, but maybe we can try a few experiments to see if karma distribution changes for topics (e.g. by LDA)
% * Paper issues:
% ** Naturally biased, since it _defines_ influence as this. Karma is directly observable so we
%    can see what actually corresponds to karma


\section{Project Proposal}
\bibliography{proposal}{} \bibliographystyle{acl2012}

\end{document}

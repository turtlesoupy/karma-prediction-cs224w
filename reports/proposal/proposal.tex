\documentclass[11pt]{article}
\usepackage{acl2012}
\usepackage{geometry}
\usepackage{times}
\usepackage{latexsym}
\usepackage{amsmath}
\usepackage{multirow}
\usepackage{url}
\newgeometry{margin=2.85cm}
\DeclareMathOperator*{\argmax}{arg\,max}
\setlength\titlebox{6.8cm}    % Expanding the titlebox

\title{{\small CS224W Final Project} \\ Proposal}
\author{Thomas Dimson \\
  {\tt tdimson@cs.stanford.edu}
  \\\And
  Milind Ganjoo \\
  {\tt mganjoo@cs.stanford.edu}
}
\date{}

\newcommand{\titlecite}[2]{``#1''~\cite{#2}}

\begin{document}
\maketitle

\section{Introduction}
% Make clear the difference between karma and influence (karma directly measurable
% but unknown how it correlates)

\section{Reaction Paper}
There is a wealth of existing literature on identifying and quantifying ``influencers''
on social networks such as Twitter. Although it remains to be seen whether \textit{karma} and
\textit{influence} are related, we believe that many of the techniques used in these papers
could be used to inform our analysis of user karma.

\subsection{Quantifying Influence on Twitter}
In \titlecite{Everyone's an influencer: quantifying influence on twitter}{bakshy2011everyone}, 
the authors define \textit{influence} as the ability to generate cascades containing
URLs on twitter. Technically, they begin by examining the first tweet (seed) containing the URL and
then measure how the URL propagates through the follower graph of the seed user. To measure their
success, they train a model that attempts to predict cascades based on attributes of the user
(e.g., number of followers or the number of tweets) and their past ability to generate cascades.
Unsurprisingly, the past ability to predict cascades was the best predictor of future performance.

A big take-away from this paper is that local-features, those from a user's immediate followers,
are better indications of influence than global ones (wider properties of the graph). The authors
also attempt to see if cascades are affected by the \textit{content} of the tweets by categorizing the 
tweeted URLs using mechanical turk. They admit that this categorization did not help in their
prediction task. In our work,
it seems worth investigating whether \textit{karma} of repliers is a big indicator of the karma
of the person posting and whether their are ``karma cliques'' in the graph. 

We take issue with two parts of the paper: first, the authors directly define \textit{influence} 
as the ability to generate cascades. Influence is a nuanced concept, and it seems that cascades
may only be a small factor in it. We also think that the authors' analysis of content in cascades
is naive - depending on humans to evaluate content made their dataset small. Our investigation
could include a larger scale analysis of content in the form of a topic modelling algorithn
such as \titlecite{Latent dirichlet allocation}{blei2003latent} (LDA).

The authors' task differs from our task because unlike karma, influence is not necessarily
a directly observable quantity. That said, we believe that the approach of the paper provides a direction
for us: attempting to come up with a model of karma and then quanitfying our success
by our ability to predict it.

\subsection{Twitterrank}
In \titlecite{Twitterrank: finding topic-sensitive influential twitterers}{weng2010twitterrank}, 
some nifty jazz occurs.
% ROUGH NOTES
% * Paper comes up with a new ranking algorithm it claims corresponds to influence on twitter
% * Claims PageRank ignores the /interest/ of twitter
% * Aggregates tweets of a user into a single document and performs LDA on the resultant docs
% * Idea: followers are likely to have similar probability distributions over topics (
%   as measured by KL-divergence)
% ** Validates this hypothesis on their dataset emphatically
% * Invents a modified version of topic sensitive page rank that has transition probabilities
%   to all nodes affected by their topic similarity.
% * Evaluation: remove follower/target relationship, run algorithm, see if they can predict the relationship
% ** Algorithm performs very well in this task
% * They quantify general influence as a weighted sum over topic-specific influences
% * Take aways:
% ** It seems plausible that karma is influenced by the primary subject area of the poster
% ** We might require multiple models for karma depending on topic
% ** We should validate the hypothesis that repliers and posters tend to have similar topic distributions
%rguujkngddfrkcldkvjejfkvtcltkvt ** It is possible that karma is related to PageRank, and refined with topic sensitive PageRank
% * Paper issues:
% ** Evaluation set was seeded by 1,000 top singaporian twitter users, not randomly. Prone to implicit bias
% ** Only 6748 people in the evaluation set - seems very small

\subsection{False indicators of influence}
In \titlecite{Measuring User Influence in Twitter: The Million Follower Fallacy}{cha2010measuring}, 
some additional nifty jazz occurs.
% ROUGH NOTES
% * Paper compares three measures of influence: indegree, retweets and mentios.
% * Observes how popular news topics spread by the three typeso f influential users.
% * Questions the traditional view that only a few select people have an instrinsic quality that makes them influential. i.e anyone can "make it" with the right amount of focus or the right set of circumstances.
% * Take away 1: the three measures mean different things, e.g followers/indegree indicates "popularity", retweets represents content value, and mentions indicate name value. Our objective is similar -- we intend to study what aspect of influence karma score actually measures.
% Note 1: can we sift out users with high karma values but who are just popular because they've been around long, and not necessarily because they're useful? How do you measure "value" of their contributions? Can we study the nature of their relationships with others? (e.g. are they perceived positively by other high-ranked members? or is there a lot of antagonism/arrogance?)
% * Note 2: can we measure evolution of karma with time? Don't want "15 seconds of fame" people who've since then retained their influence, because karma scores are usually static.
% * Methodology: rank correlation coefficient (need to discuss this in a little detail) to compare different measures of influcence
% * Findings: retweets usually prominent among content aggregation services (content value), mentions among celebrities (name value)
% * Second study: does influence change across topic genres? (ie is a person who posts only about one theme more or less influential than a person who posts about everything?) According to them, generalists tend to have more influence, but it would be nice to use our LDA topic stuff to segment users and try performing influence calculations in those separate contexts.



We'll probably cite \titlecite{Topic-sensitive pagerank}{haveliwala2002topic} too. And what kind of a crazy
paper doesn't cite 

\section{Project Proposal}
\bibliography{proposal}{} \bibliographystyle{acl2012}

\end{document}

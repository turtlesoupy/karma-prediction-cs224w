\documentclass[11pt]{article}
\usepackage{acl2012}
\usepackage{geometry}
\usepackage{times}
\usepackage{latexsym}
\usepackage{amsmath}
\usepackage{multirow}
\usepackage{url}
\newgeometry{margin=2.85cm}
\DeclareMathOperator*{\argmax}{arg\,max}
\setlength\titlebox{6.8cm}    % Expanding the titlebox

\title{{\small CS224W Final Project} \\ Proposal}
\author{Thomas Dimson \\
  {\tt tdimson@cs.stanford.edu}
  \\\And
  Milind Ganjoo \\
  {\tt mganjoo@stanford.edu}
}
\date{}

\newcommand{\titlecite}[2]{``#1''~\cite{#2}}

\begin{document}
\maketitle

\section{Introduction}
Our project blah blah blah

\section{Reaction Paper}
Summary, critique, brainstorming.


\subsection{Quantifying Influence on Twitter}
In \titlecite{Everyone's an influencer: quantifying influence on twitter}{bakshy2011everyone}, 
some stuff happens.

% ROUGH NOTES
% * Paper observes cascade / diffusion events to quantify "influencers"
% * Defines influencers as those who disproportionally impact the spread of information
% ** We could investigate whether this corresponds to karma, if our network has cascades
% * Methodology: observe the origin of URLs, then track them as they are tweeted by followers
% * Trains model for cascades based on followers, tweets,  friends, past influence. Past influence most predictive
% ** Finds local features (direct followers) are more important than global. Could try to see if karma is influenced
%    by people close by
% * Also tries to find out the role of /content/ in cascades by using mech turk to classify the URLs
% ** Did not get good results, but maybe we can try a few experiments to see if karma distribution changes for topics (e.g. by LDA)
% * Paper issues:
% ** Naturally biased, since it _defines_ influence as this. Karma is directly observable so we
%    can see what actually corresponds to karma

\subsection{Twitterrank}
In \titlecite{Twitterrank: finding topic-sensitive influential twitterers}{weng2010twitterrank}, 
some nifty jazz occurs.
% ROUGH NOTES
% * Paper comes up with a new ranking algorithm it claims corresponds to influence on twitter
% * Claims PageRank ignores the /interest/ of twitter
% * Aggregates tweets of a user into a single document and performs LDA on the resultant docs
% * Idea: followers are likely to have similar probability distributions over topics (
%   as measured by KL-divergence)
% ** Validates this hypothesis on their dataset emphatically
% * Invents a modified version of topic sensitive page rank that has transition probabilities
%   to all nodes affected by their topic similarity.
% * Evaluation: remove follower/target relationship, run algorithm, see if they can predict the relationship
% ** Algorithm performs very well in this task
% * They quantify general influence as a weighted sum over topic-specific influences
% * Take aways:
% ** It seems plausible that karma is influenced by the primary subject area of the poster
% ** We might require multiple models for karma depending on topic
% ** We should validate the hypothesis that repliers and posters tend to have similar topic distributions
% ** It is possible that karma is related to PageRank, and refined with topic sensitive PageRank
% * Paper issues:
% ** Evaluation set was seeded by 1,000 top singaporian twitter users, not randomly. Prone to implicit bias
% ** Only 6748 people in the evaluation set - seems very small

We'll probably cite \titlecite{Topic-sensitive pagerank}{haveliwala2002topic} too. And what kind of a crazy
paper doesn't cite \titlecite{Latent dirichlet allocation}{blei2003latent}?

\section{Project Proposal}
\bibliography{proposal}{} \bibliographystyle{acl2012}

\end{document}
